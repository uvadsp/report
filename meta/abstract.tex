\begin{abstract}
Every country is affected by crime, and so is the Netherlands. In order to battle drugs related organized crime better, it is important for authorities, such as the police, to understand the modus operandi of criminal organizations through both academic and practical research. This study focuses on a data-driven pattern identification of drug usage and events, such as festivals and public holidays. In order to investigate this relationship, this study aims to answer the following research question: To what extent can we identify event-based drug popularity on online data resources? In the analysis, Google trends, news and Twitter data were scraped, processed and statistically analysed. The results indicated that among the three data sources, data from Google Trends best follows the increases observed in drug usage and therefore works best for pattern recognition. In addition, the analysis indicates that events that affect larger portions of the population (e.g. King’s day, New Year’s Eve, Pride, ADE) are better represented with Google Trends as opposed to events which are more specific to a smaller part of the country (e.g. Lowlands). These data analysis results were visualized in a dashboard which is made publicly available. In future work, the robustness and effectiveness of the system could be further enhanced by considering the expansion of data sources and processing methods, adaptation to accommodate longer events, integration of real-time API data, and conducting comprehensive usability testing.
\end{abstract}