\section{Related Work}

The collection and utilization of large amounts of data is a popular resource nowadays and is used in a variety of disciplines. [bron] The use of these large amounts of data, also known as big data, combined with the use of data analysis is having a major impact on the social sciences and humanities in general. [bron] Chan and Moses' research shows that it has a particularly large impact for the specific field of criminology. [bron] 

\subsection{Predictive policing}

Predictive policing is a term used for predicting certain behaviors or trends based on analyzing various data for law enforcement. [bron] Authorities use predictive policing to predict crime in order to prevent the criminal activity instead of reacting to the crime that already occurred. For this reason, the domain and effects of predictive policing have been examined for a long time. 

However, not all research agrees that predictive policing is a valid method. According to The New York Times' debate, predictive policing is a very effective way when it comes to predicting criminal behavior, but still contains many improvements in terms of ethicality. [bron] In contrast, Hardyns' research says that the analysis of such predictions has indeed shown its worthiness for different predictive systems in different areas, but exactly because it is a new development in the field of criminology, little is yet known whether it is an effective way for law enforcement. [bron] 

Several studies have been conducted that have investigated whether Google trends can be a possible resource for predicting particular trends. Previous research has shown that Google Trends can be used as a predictor in different fields. Such as the healthcare industry, where Google Trends can help support the prediction of the outbreak of seasonal influenza and COVID-19 [bron], [bron]. Another study by Kassraie et al. used Google trends data in combination with Twitter data to predict the popularity vote of the 2016 presidential election in the US. They concluded that the combination of these social media platforms could be a mirror for the public opinion on political events [bron].

In the study by Perdue et al. it appears that Google trends can be a possible predictor for drug abuse trends. [bron] This is supported by the research of Gamma et al. which investigated whether there is a comparison between time trends of Google search interests and offenses committed in relation to the drug called Methamphetamine. From this study, it was found that law enforcement could indeed use the Google search feature as a possible predictor of Methamphetamine-related crimes. [bron]