\section{Discussion}
The research suggests that the Google Trends data has the highest similarity to the drug usage at events, compared to the Twitter data and NOS articles. However, there are some limitations that need to be considered when interpreting the results. First of all, the amount of data from the NOS articles was small. This resulted in only having a few articles that mentioned one of the drugs per week. The data was transformed into relative scores between hundred and zero, which caused the NOS articles scores to have very large differences between the data points. If the data size of the news articles were bigger, the timeline would probably have contained less peaks and therefore might have had better results on the performed proportion tests.

Another limitation of this research is the usage of weekly data. The timeframe of the data used is nine years. For a time period of more than five years Google Trends only shared weekly data, which obliged the research to use weekly data for all data sources. However, the events used did not have a timeframe of a week and therefore could fall on different days of a week. This could have affected the results, as an event that falls on a Monday or Sunday is more likely to also cause an increase of drug usage on another week than the one tested for.
Moreover, in this research no tests were performed on delays shown in the data. For example, the Twitter data did not once show a significant difference of peak proportion for a week of an event. However, it could be discussed that drug events tend to be trendy on Twitter one or two more weeks after the actual date of the event.

Finally, one of the main limitations of this research is the lack of user availability testing. During the research no opportunity for meeting the end users of the dashboard has been given. Therefore, the usability of the prototype may have some shortcomings.

