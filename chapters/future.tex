\section{Future work}

\subsubsection{Expand data sources}
A subset of drugs and years was used for our research. To allow for more exploration and filterability the dashboard could be expanded with more datasets. Especially more relevant drugs (upcoming drugs, non-legal substances, NSPs) could be added as filter options. We also currently only use NOS news data but other more popular dutch news sources (e.g. RTL nieuws, Nu.nl) could be scraped to give a more accurate representation of the popularity of drugs mentioned in news articles. This will both improve the quality of the dataset by more accurately calculating a relative score as well as quantitify by gathering multiple sources and aggregating.

\subsubsection{Real-time API data}
Currently the dasboard relies on exported data that is then loaded into the web visualization. As a futher enhancement for the prototype and to make it more dynamic is to have the dataset exposed as an API which the dashboard can then fetch getting up-to-date real-time data. In this beta version datasets need to be added manually to the GitHub repository.

\subsubsection{Usability testing}
Further usability testing needs to be done to validate the User Experience (UX) and User Interface (UI) of the prototype to more accurately represent the needs of the police analyst. Basic user testing was done on a small group of people and feedback from the client was incorporated. But still, a lot of assumptions about the user have been made. User testing the live version of the prototype to a larger user base will uncover hidden problems. The feedback from the users would further validate the workings of the prototype.

In general further research is required to conclude that online data-gathering tools fare good indicators for predicting the popularity of specfic drugs.