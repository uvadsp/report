\section{Future work}

In future work, it will be important to consider several key factors to enhance the robustness and effectiveness of the system which are described in more detail below. Firstly, the expansion of data sources and processing methods will be crucial in order to obtain a more comprehensive understanding of the data being analyzed. This will allow researchers to effectively capture the full range of relevant information and to identify key trends and patterns more accurately. Secondly, the system must be adapted to accommodate longer events, which will require significant advancements in both data processing and storage capabilities. Thirdly, the integration of real-time API data will play a critical role in ensuring that the system remains responsive to changing conditions and can provide up-to-date information. Finally, it is essential to conduct comprehensive usability testing to evaluate the system's performance and user experience and to identify opportunities for improvement. By taking these considerations into account, the system continues to evolve and meet  the needs of all stakeholders in a scientifically rigorous and effective manner.

\subsubsection{Expand data sources}
A subset of drugs and years was used for our research. To allow for more exploration and filterability the dashboard could be expanded with more datasets. Especially more relevant drugs (upcoming drugs, non-legal substances, NSPs) could be added as filter options. Besides, currently, only NOS news data is used but other popular dutch news sources (e.g. RTL nieuws, Nu.nl) could be scraped to give a more accurate representation of the popularity of drugs mentioned in news articles. This will both improve the quality of the dataset by more accurately calculating relative scores, as well as expand the results by gathering multiple sources and aggregating.

\subsubsection{Expand data processing}
At the moment the data was aggregated and tested as described in the Methodology section. Future work can expand the research by incorporating additional approaches to data processing, like sentiment analysis and Word2Vec. As explained in the Data Methods section, the (unstructured textual) data used offers the possibility of using these techniques and code scripts are already prepared as part of the work in this research. However, due to a time limitation, the output has not been incorporated in the results. More specifically, sentiment analysis would be useful to incorporate in the dashboard as an indication to the dashboard users about the sentiment of searches, tweets and news articles that relate to the drugs of interest. This can reveal additional important information about user behaviour that the police could use for investigation. Word2Vec can be used for synonym extraction. At the moment the data is based on the exact forms of the drug names (with variations of upper- and lowercase where necessary). Incorporating synonyms can help extract broader information on the same topic.

\subsubsection{Adaptation to longer events}
At the moment the data is aggregated weekly, which captures the effects that single-day events have within the same week before and after they happen. Future research can focus on replacing the aggregation of week number with an aggregation on a 7-day period, regardless of whether those days are in the same calendar week. This would more flexibly catch the effects of events that happen at the beginning or the end of a calendar week. Moreover, the time range around events that happen in more than one day (e.g. festivals) could be adjusted to a number of days before the start and after the end of the event. The limitation of doing this in the current research was the lack of possibility to aggregate Google Trends data on a non-calendar week. 

\subsubsection{Real-time API data}
Currently, the dashboard relies on exported data that is then loaded into the web visualization. A further enhancement for the prototype and to make it more dynamic is to have the dataset exposed through an API which the dashboard can then fetch getting up-to-date real-time data. In this beta version datasets need to be added manually to the GitHub repository.

\subsubsection{Usability testing}
Further usability testing needs to be done to validate the User Experience (UX) and User Interface (UI) of the prototype to more accurately represent the needs of the police analyst. Basic user testing was done on a small group of people and feedback from the client was incorporated. But still, a lot of assumptions about the user have been made. User testing the live version of the prototype to a larger user base will uncover hidden problems. Direct feedback from the users would further validate the workings of the prototype.

In general further research is required to conclude that online data-gathering tools fare good indicators for predicting the popularity of specific drugs.