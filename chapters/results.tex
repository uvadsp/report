\section{Results}

\subsection{Selection of events}

In the Netherlands, several festivals and parties are organized on a large scale that attracts a lot of visitors. Because a very small amount of numbers are known about drug use in the Netherlands at certain events and holidays, it is hard to say at which events drug use is more popular than others. For this reason, the research focused on news articles showing high levels of drug use at various events. In order to conduct a proper analysis within the time frame, five events were chosen.  

\subsubsection{Unpopularity of drug use}

A small survey of drug use at various events revealed that not all major events in the Netherlands involve high levels of drug use. A remarkable aspect is that at De Zwarte Cross, the largest festival in the Netherlands with 220 thousand visitors, hard drugs are not popular among the audience. De Zwarte Cross is originally a cross race that lasts 3 days combined with music and performances by various bands. At De Zwarte Cross a lot of alcohol is consumed in particular \cite{Sprikkelman}. In addition, it appears that Carnival, a multi-day festival celebrated in the south of the Netherlands, also mainly consumes a lot of alcohol and has much less interest in hard drugs \cite{Kerstens}. 

It can be explained that drugs do not have a major presence at these large events because drugs are mainly used during dance festivals where techno and house music is dominant. According to \cite{Coppens}, 70\% of dance festival visitors use drugs. Both at De Zwarte Cross and during Carnival, this style of music is not played very frequently and this could be a reason for the unpopularity of drug use during those events.

However, the Amsterdam Dance Event (ADE) appears to be a popular event for drug use \cite{RTL}. During this multi-day dance festival that consists of multiple parties spread across multiple locations in Amsterdam, a lot of techno and house music is played. This would support the explanation of drug use and certain parties and festivals. 
Another festival where drugs appear to be widely used is Lowlands \cite{NOS}. This festival lasts one weekend at the beginning of August and various artists around the world perform at this festival. 

Besides these big festivals, many parties and smaller festivals take place during certain holidays in the Netherlands. For example, Koningsdag, one day every year when the Dutch celebrate the king's birthday, is a real festivity. According to several articles, many people also use drugs during this day \cite{Boer}. Not only during Koningsdag many people appear to use drugs also during Pride Amsterdam and New Year's Eve. Pride Amsterdam is a multi-day celebration with different events organized in several cities in the Netherlands. During Pride, many Dutch people celebrate freedom of sexuality. This party has a gay cultural character and is known to involve many different types of drugs \cite{Beusekamp}.

Overall, the five events identified based on this news-article survey where drugs are commonly used are;

\begin{enumerate}
  \item Amsterdam Dance Event
  \item Lowlands
  \item Kingsday
  \item Pride Amsterdam
  \item New Year's Eve
\end{enumerate}

\subsubsection{Established time period}

So for the research in this report, those five events are used as events with a high level of drug use by the visitors. Furthermore, the time period of each event is set to 1 week in which the event takes place. In this way, the preparations for the event on the popularity of drugs are also included in the research. To carry out the research as automatically as possible, it is difficult to set a different time period for different events, such as ADE, which itself lasts a week, in which the preparations will also be included. Therefore it is decided to consider all events as one-week events and Future Work will describe how the research could be adapted for longer events. 

\subsection{Selection of drugs}

For this research in which drugs and specific events in the Netherlands are analyzed, it is important to determine the specific drugs under investigation. Because of the limited timeframe for this research and the variety of drugs in the Netherlands nowadays, it is chosen to select three types of drugs. If the outcomes of the three drugs contain interesting results or will serve other potential purposes, the research could also be conducted for multiple drugs and displayed in the final dashboard. More on this will be discussed in Future Work.

The three drugs that are used in this study and are visualized in the resulting visualization are based on the statistics from the National Drug Monitor. The National Drug Monitor (NDM) is a Dutch organization that collects and compiles all data on drug use, the drug market and drug-related crime of all ages in the population through different monitors. The purpose of this institution is to provide a clear representation of all the data monitored in the Netherlands related to drugs \cite{Trimbos}. 

One of those representations is given in Figure X. This figure shows for each drug how much it has been used among the Dutch population in recent years. This figure shows that the most commonly used drug, with a significant difference, is cannabis. In addition, it can be seen that XTC is a commonly used drug in the Netherlands and cocaine is in third place. As can be seen in the figure, the line representing the drug Nitrous oxide (Lachgas) is interrupted because no results are known about this drug before that time. For that reason, this drug is not included in the research at all. The other drugs visible in the figure that are not so often used are heroin, Amphetamine, Paddo’s, LSD, and GHB.

To investigate the popularity of drugs it is more interesting for the police to look at hard drugs that are illegal to possess and use than at the tolerated drugs for which the police cannot intervene. For this reason, although the soft drug cannabis is the most commonly used drug in the Netherlands, it is not investigated further.

As concluded above, the two most commonly used hard drugs in the Netherlands are XTC and cocaine. Because these are the most used hard drugs they are expected to be the most popular drugs in the Netherlands which can be interesting for the research. 

Besides that, it is chosen to include a less-used hard drug in the research. This decision is made to investigate whether less frequently used hard drugs are also less frequently searched on Google and appear in news articles than the commonly used hard drugs and whether those drugs are related to specific events (and how) or not. According to figure X from the National Drug Monitor, GHB is a rarely used drug in the Netherlands and therefore GHB is examined as the third drug. along with cocaine and XTC in this report. 

So for this research, three drug types that are specified as cocaine, XTC and GHB are investigated, because it provides sufficient results for the Police Academy and fits within the time span for this research. 

\subsection{Proportion test results}

To answer the main research question and the two subquestions ‘How could we identify drug popularity at events?’ and ‘What are the differences between the specific events and drugs?’, proportion tests have been performed on Google trends, Twitter and NOS data for the three types of drugs. Thus, the mean proportion of peaks for all weeks from 2014 until 2022 has been compared to the proportion of peaks for specific drug events. For each test, the null hypothesis (H0) stated: the proportion of peaks observed in drug-related event weeks is not significantly larger than the proportion of peaks in regular weeks. The alternative hypothesis (H1) for each test stated: drug-related event weeks have a significantly higher proportion of peaks compared to the regular weeks in the year.

\subsubsection{XTC}

The results show that for the drug XTC and the Google Trends data, King’s Day (p < .001), New Year’s Eve (p < .001) and Amsterdam Dance Event (p = .009) have rejected H0. However, Lowlands (p = .700) and the Amsterdam Gay Pride Event (p = .183) did not show a significant difference. Therefore, three out of five events have a strong significant difference with the mean proportion of Google trends data for the drug XTC.

However, statistics for the Twitter data on the drug XTC show that for King’s Day (p = .977), New Year’s Eve (p = .461), Amsterdam Dance Event (p = .461), Lowlands (p = .211) and the Amsterdam Gay Pride (p = .760) H0 has not been rejected. Thus, Twitter data did not show any significant difference between the mean proportion of all weeks and the proportion of the weeks of all events.

Furthermore, the results show that the NOS articles also did not have a significant difference in peak proportions. Since, King’s Day (p = .264), New Year’s Eve (p = .887), ADE (p = .544), Lowlands (p = .544) and Pride Amsterdam (p = .887) all did not reject H0. Thus, the drug XTC proved to only have a strong significant relationship with the Google Trends data and the events of King’s day, New Year’s Eve and ADE.

\subsubsection{Cocaine}

For cocaine, the results do not show any significant outcomes for the Google Trends data. As King’s Day (p = .067), New Year’s Eve (p = .435), Amsterdam Dance Event (p = .205), Lowlands (p = .205) and the Amsterdam Gay Pride (p = .731) do all show p-values higher than 0.05. However, the p-value of King’s Day (p = .067) is only slightly higher than the critical p-value. 

	Moreover, the results show that for Cocaine and the Twitter data, King’s Day (p = .200), New Year’s Eve (p = .745) and Amsterdam Gay Pride Event (p = .973) have not rejected H0.  Amsterdam Dance Event (p = .055) and Lowlands (p = .055) show p-values quite low. Nevertheless, the p-values of ADE and Lowlands are not high enough to reject H0. Therefore, Cocaine does also not show a significant difference between the mean proportion of the Twitter data and the proportion of the events. 

	Lastly, statistics do not show that Cocaine and the NOS articles have a difference in the proportion of peaks with the proportion of peaks of any event. For King’s Day (p = .561), New Year’s Eve (p = .373), Amsterdam Dance Event (p = .676), Lowlands (p = .373) and the Pride Amsterdam (p = .952) all did not reject H0. Thus, cocaine does not show any significantly higher proportion of peaks between the different data sources and the events.

\subsubsection{GHB}

The test results for the drug GHB and the Google trends data show that for the event of King’s Day (p = .011) it rejects H0. New Year’s Eve (p = .205), Amsterdam Dance Event (p = .067), Lowlands (p = .970) and ADE (p = .067) do not show any significant results. However, the p-value of both Amsterdam Dance Event and Amsterdam Gay Pride is only slightly above the critical value. Therefore, for the drug GHB, Google Trends only show a significant difference in peaks for King’s Day.

Additionally, the statistics demonstrate that for GHB, Twitter does not have a significant difference in peaks with the events. Since, King’s Day (p = .472), New Year’s Eve (p = .979), ADE (p = .219), Lowlands (p = .472) and Pride Amsterdam (p = .219) did all not prove to be significant.

Finally, the NOS articles also do not show any significant results for the drug GHB. King’s Day (p = .516), New Year’s Eve (p = .516), ADE (p = 1.00), Lowlands (p = .516) and Pride Amsterdam (p = .216) all do not reject H0. Thus, out of all data sources, the drug GHB only shows a significantly higher proportion of peaks for Google Trends during King’s Day . 

In conclusion, the test results show that the subquestion ‘How could we identify drug popularity at events?’ can be answered by stating that Google Trends, with a score of four, has the highest number of significant proportions of peaks for the events in total. In contrast with Twitter and the NOS articles where none of the events demonstrated to have a significant proportion of peaks for all types of drugs tested. 

Concerning the last subquestion ‘What are the differences between the specific events and drugs?’, there is a difference found between the specific events. The statistics of the proportion tests demonstrated that the events Lowlands and the Amsterdam Gay Pride never reject H0 for any type of drug tested, while King’s Day, New Year’s Eve and ADE did reject H0 for the drug XTC. Additionally, King’s Day proved to have a significantly higher proportion of peaks for the drug GHB as well.


\subsection {Prototype (dashboard)}\hfill

\noindent Based on these requirements and the datasets a custom web-based visualization was created to allow the analyst to visualize explore, compare and filter the datasets. When filtering the charts update in real-time with keyframed animations. Screenshots and a live version of the prototype are included in appendix 2a. For this prototype demo, the user is logged in with a default user account which is shown in the sidebar navigation. From there the user has four different overview pages to navigate to. 

\subsubsection{Events overview page}
The first is an events overview page which shows a line chart with on the x-axis the weekly numbers per year and a relative score from 0 to 100 on the y-axis for our three datasets, the Google Trends, NOS News data and Twitter tweets. On the left bottom is a legend with a list of drug-related events. The user can click on an event which will highlight the week in which the event occurred in the line chart to more clearly visualize peaks in the data which indicate event-based drug popularity. A small description of the event, the most popular drug and event dates also are shown. The bottom-right shows the Google Trends data in a relative score overall 5 years of our dataset. This shows more clearly shows the lack of peaks of events in covid pandemic years.

\subsubsection{Region overview page}
The second page is a choropleth map page showing the regions of the Netherlands. Each region is colour coded using a linear colour scale based (blue interpolation) on the relative score of Google searches for that region. On this page, the user also has the option to focus on years and drugs but also on specific event dates. This, for example, shows that in Flevoland when the Lowlands event is happening there are a lot of searches in that region for the drugs XTC.

\subsubsection{Related queries page}
The third page is the related queries page which shows a line chart for our three chosen drugs as filter options. The y-axis is a relative score from 0 to 100 and the x-axis are the week numbers for our specific year. The user can filter between different years. On the bottom are three polar charts which show the related drugs people are searching for with a relative score and a subset of the drugs in the legend. For example, people who search for cocaine also search for crack very often. It also shows that before 2019 GHB was not searched for by many people. It increased in popularity after that year.

\subsubsection{Settings page}
A settings page is also included. This page shows the different datasets used in the dashboard and a download button which allows the user to download a specific dataset either as .csv or .json to store on their computer.

\subsubsection{Web application}
The prototype is a web-based application using web standards and open-source software and libraries. It being web-based allow the dashboard to run the operating system independently. Only a web browser installed on the user's device is required. The application will mainly be used in a desktop environment by the analyst so the dashboard is not fully responsive and not mobile-optimized. The web application is created with the open-source front-end framework Svelte \footnote{https://svelte.dev} and UI framework SvelteKit which allows the application to be built in components, each chart is rendered separately making it more efficient to add functionality (e.g. add datasets, render different chart types) in the future but also makes the dashboard performant when more data and charts are added since Svelte already pre-renders the page offloading work from the browser. Working in components and with frameworks such as Svelte allows future web developers to get up and running fast and add more functionality in a progressively enhanced manner. Svelte can be downloaded as a module (package) from NPM \footnote{https://www.npmjs.com} and uses the JavaScript back-end run-time Node.js \footnote{https://nodejs.org/}. 

For the charts the JavaScript charting library Chart.js \footnote{https://www.chartjs.org} is integrated into the components which allows charts to be rendered in HTML5 Canvas without much configuration. With Chart.js you can add a specific dataset and the scales of the axis will automatically change accordingly to the scale defined by the data. The filter options and updating of the charts are more custom, it uses JavaScript utility functions to allow the data to be pre-processed and have only the data changed not the scales of the whole charts. For the map page, an additional Chart.js Geo Plugin is used to render the Chloropleth map. It uses a TopoJSON file to render the regions of the Netherlands and filters the properties within that .json file to render a score for each region.

The source code for the dashboard is published open-source on GitHub using the MIT license. A live version of the dashboard is continuously deployed on the hosting platform Netlify \footnote{https://www.netlify.com}. Corresponding links can be found in appendix 2b.

\subsubsection {Usability Testing}\hfill

During the project, we on several occasions tried to arrange meetings with analysts from the police academy but due to planning problems, it proved difficult to arrange a meeting at the location of the police academy to talk to actual analysts. This did not prevent us from doing some initial and basic user tests on the beta versions of our dashboard. 

The first version of our dashboard we tested among students that were part of our working group. A summative usability test was performed with a total of six persons who were not involved in the development of the project, which is sufficient to identify basic usability problems \cite{nielsen}. We observed and noted where they encountered problems using the Thinking Aloud method. 

After incorporating this feedback iteratively in the prototype we did a second usability test with the Expert Review method. Two senior lecturers of UX/UI from the Faculty of Digital Media and Creative Industries at the Amsterdam of University of Applied Sciences reviewed the interface of the prototype and navigated through all functionality.

Examples of incorporated feedback from the test were improving the visual design mainly by adding secondary colours to have sufficient colour contrast and adding descriptions and labels of the icons in the navigation to clarify the meaning of the pages. In addition,  legends were adjusted to make clear at what scale the charts show.