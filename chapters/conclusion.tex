\section{Conclusion}
The research presented in this paper focused on discovering creative data sources that can be of help to the Police Academy when performing investigations on cases relating to drug usage. Three data sources were researched: Google Trends, Twitter and Dutch news (NOS). The hypothesis was that the data from these sources is representative of drug user behaviour in The Netherlands.

The research consisted of multiple steps. First events known for increased drug usage were identified in order to serve as a proxy for increased drug usage. The dates and weeks of each event were identified between 2014 and 2022. Afterwards, drugs of interest we chose based on popularity in events and in The Netherlands overall. Then, using the same time range (2014-2022), the data sources Google Trends, Twitter and Dutch news (NOS) were scraped to generate data: the content of searches, tweets  and articles respectively. The data was aggregated and ordered chronologically to perform the analysis. Finally, the collected data was tested in order to discover whether an increased trend of the data was observed consistently during the weeks of the selected events compared to the rest of the weeks. A proportion test was performed for each combination of drug, event and data source. 

The results indicated that among the three data sources, data from Google Trends best follows the increases observed in drug usage. The null hypothesis was rejected the most for drugs that are popular during events (XTC, GHB). The test failed for cocaine, which could be explained by this drug being popular over the entire year. In other words, increased usage is not observed for cocaine during events, and the data from google trends also does not show significant peaks. 

An additional conclusion from the results was that events that affect larger portions of the population (King’s day, New Year’s Eve, Pride, ADE) are better represented with Google Trends as opposed to events which are more specific to a smaller part of the country (Lowlands). 
