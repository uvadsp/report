\section{Introduction}

Every country is affected by crime, and so is the Netherlands. According to the Dutch centre for crime prevention and safety statistics (CCV), over 90\% of all organized crime are related to illegal drugs \cite{CCV}. The Netherlands owes this percentage to its prominent role in the international drug trade \cite{mcdermott}. Since problems surrounding drugs have increased considerably in recent years, authorities have prioritized and focused on identifying and tackling drugs related behaviour \cite{eski, ferwerda}. 

In order to battle drugs related organized crime better, it is important for authorities, such as the police, to understand the modus operandi of criminal organizations through both academic and practical research. One of the organizations that focuses on the discovery of new modus operandi is the Police Academy (PA). In cooperation with the University of Amsterdam, the PA wants to focus on data-driven research to identify drug-related patterns in the Netherlands. 

Previous research within the narcotics space focused on the supply side of the illegal drug trade \cite{paoli, magliocca}. This domain focuses on understanding the production, distribution, and trafficking of illegal drugs. It includes research on the modus operandi used by criminal organizations to produce and transport drugs, as well as the routes and networks they use to distribute them. This type of research is important for law enforcement agencies as it helps them identify and disrupt the operations of drug traffickers \cite{mcdermott}.

Another area of research that is also present in prior literature is the demand side of illegal drugs. This domain focuses on understanding the consumption and impact of illegal drugs on individuals, politics, and society \cite{riley, flores}. It includes research on the psychological and social factors that contribute to drug addiction, and the health and social consequences of drug use \cite{gonzalez, sesnie}. 

These research domains are important for public health agencies and policymakers as it helps them understand the extent of the drug problem and develop effective strategies for addressing it. However, most of these studies are less relevant for authorities such as the PA since they focus less on quantitative data-driven methods to identify patterns in society. Therefore, this study focuses on a data-driven pattern identification of drug usage and events, such as festivals and public holidays. In order to investigate this relationship, this study aims to answer the research question:
\textit{"To what extent can we identify event-based drug popularity on online data resources?"}. To answer this research question, the following sub-questions were formulated:

\begin{itemize}
  \item Which Dutch events are indicative for drug popularity?
  \item What drugs are used during events in the Netherlands?
  \item Which online data resources are relevant to identify drug popularity at events?
  \item How could we identify drug popularity at events?
  \item What are the differences between the specific events and drugs?
  \item How could we visualize event-based drug popularity on online data resources?
\end{itemize}

The sub-questions are relevant to the main research question as they provide a more detailed and specific understanding of the topic. The subquestion "Which Dutch events are indicative for drug popularity?" helps to narrow the scope of the research since it provides a starting point for identifying which events may be associated with drug use. Second, the subquestion "What drugs are used during events in the Netherlands?" is important for identifying which specific drugs are of interest for the research. "Which online data resources are relevant to identify drug popularity?" and "How could we identify drug popularity at events?" are crucial for understanding the methods and data sources of this research. "What are the differences between the specific events and drugs?" helps to identify any patterns or trends in drug popularity that may vary depending on the specific event or drug being considered. The last subquestion “How could we visualize event-basedx drug popularity on online data resources?” is relevant since it allows us to transform our research data into a user-friendly dashboard to create insights for the PA.

In order to answer the research questions, we first present the theoretical state of affairs, followed by the methodological set-up, after which the results are presented. In the final sections, the most important findings are concluded and limitations are discussed, followed by recommendations for future work.
