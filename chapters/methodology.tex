\section {Methodology}

The approach to answering the research questions will be described in this section. The methodology consists of four parts. The first part describes the methods and literature that was used to select drugs and events. The second part describes the data pipeline which includes the data selection, data collection, data processing and data privacy. The third part describes the statistical metrics that were used to analyze the data. The last part covers the methods that were used to convey and visualize the data into the dashboard. 

\subsection {Methods for identifying drugs and events}

\subsubsection {Method for drug selection}

To determine the different types of drugs which are used in the study, several institutes that collect data on drugs in the Netherlands were analyzed. The largest institution in the Netherlands that is related to drugs is the Trimbos Institute (TI). This national organization conducts research on the mental health of the Dutch people with a focus on the use of alcohol, tobacco and drugs. They involve all age groups in society and therefore cover the entire life cycle of citizens. This institute releases various analysis on alcohol, tobacco and other drug use in the Netherlands which includes reports on how to address these problems by naming prevention, education and policies. 

In addition, The TI tracks drug-related developments through various monitoring systems. The most important monitor is the National Drug Monitor (NDM). The NDM collects and compiles all data on substance use, the drug market and drug-related crime of all ages in the population. This institution aims to provide a representation of the figures known in the Netherlands related to drugs. Using both data from TI and NSM, a set of specific drugs were selected for this study. 


\subsubsection {Method for event selection}

First of all, it was determined which events would provide interesting results. Because of the sub-question "Which Dutch events are indicative for drug popularity?" it is interesting to use events where drug usage occurs a lot. Because where drugs are a popular means of enjoyment, they will also be used a lot.  For this purpose, research was done on the use of drugs at different events. That way, the events served as a proxy where drug usage is known to increase.
The first method that is used is literature research. This literary research was conducted to see if there has been previous research executed on drug use for different events in the Netherlands. 
However, very little is generally known about specific drug use per event in the Netherlands. Actually there is no research done in the Netherlands about drug use at specific parties or events. Due to the fact that it is difficult to investigate because few drug users want to be open about their drug use. 
For that reason we used another method where different news articles that say something about drug use per event are examined. This involves using different news sources, such as the AD, the NOS and the Telegraaf, as well as regional news sources such as LokaalGelderland and Echt Amsterdams nieuws. 
Using these two methods, a number of events have been identified where drugs are commonly used among the public. The research and answer to the sub-question is described in Chapter X.

\subsection {Data Methods}

\subsubsection{Data selection}
Before data could be collected and analyzed, a selection of data sources has been made. When it comes to the research question "To what extent can we identify event-based drug popularity on online data resources?", Google Trends data, Twitter data, and Dutch news data represent appropriate sources of information.

Google Trends data provides a comprehensive insight into the popularity of search terms and topics in real-time. By analyzing the frequency of drug-related searches on Google, researchers can obtain a thorough understanding of drug popularity across regions and over time (Batistic, 2021). Furthermore, Google Trends offers numerous filters and visualization options, making it a user-friendly tool for data analysis.

Twitter data represents a rich source of information on public perceptions and opinions (Bian, 2016). The platform is well known for its user-generated content and real-time information, making it a valuable resource for identifying the public's views on drugs. Additionally, Twitter's hashtag and trending topic features allow researchers to quickly identify the most popular drug-related topics on the platform.

Dutch news data is a crucial source of information in the assessment of event-based drug popularity. It provides a comprehensive understanding of media representation of drug-related events, including how these events are reported and framed, and the public's perception of drug use. By analyzing news data, researchers can gain a deeper insight into the media's role in shaping public opinion on drug use (Mccombs, 2020).

In conclusion, the combination of Google Trends, Twitter data, and Dutch news data offers a diverse and comprehensive set of data sources for understanding event-based drug popularity on online data resources. These data sources provide valuable information on various aspects of drug use, such as search trends, public opinion, and media representation, making them a sound choice for answering the research question at hand.

\subsubsection{Data collection}

After data sources were selected, the data  was collected. The Google Trends data was obtained using PyTrends, a library in Python for accessing and retrieving data from Google Trends (PyPI, 2023). This method of scraping allowed for an efficient and automated process for collecting Google Trends data, enabling the researcher to obtain a large amount of data in a short period. Only searches that were done on the territory of The Netherlands were considered.

Twitter data was collected using SNscrape, a tool for scraping social media data, including tweets and user profiles (GitHub, 2023). This method of scraping Twitter data provided access to a significant amount of real-time user-generated content, allowing us to research the popularity of conversations surrounding specific drugs in the netherlands. We based on a Twitter scrape query on the time interval January 2014 - December 2022 that contained the words ‘xtc’, ‘cocaine’ or ‘ghb’. Since we are only interested in Dutch tweets, we excluded tweets that are non-Dutch. In total, the Twitter scraper collected more that 1.500.000 relevant tweets.

 The Dutch news data was obtained by downloading a news corpus from the NOS, a Dutch public broadcaster. This method of data collection allowed us to access a large amount of news articles in a centralized Kaggle repository, which was last updated at the end of 2022 (Scheijen, 2022). In total, the NOS contained more that 250.000 relevant news articles.

 \subsubsection{Data processing}

 After the selected data was collected datasets were loaded into a dataframe for processing. The pre-processing of the obtained Google Trends, Twitter, and Dutch news data was crucial in ensuring that the data was in a format suitable for analysis. To this end, the following pre-processing methods were applied:

\begin{itemize}
  \item Conversion of dates: All dates within the specified time frame were converted to the same datetime format. This standardization of the dates was essential for ensuring consistency in the data and making it easier to analyze.
  \item Calculation of week numbers: A function was created to calculate the week number of each date, as the time interval used for the analysis was "week." This function allowed for the grouping of data into weeks, making it easier to analyze trends and patterns over time.  The data was aggregated on a weekly level because that was the most granular aggregation of one of the sources (Google Trends, 2023). To have a comparable analysis, the weekly aggregation was applied to the data from all sources.
  \item Topic feature extraction: A function was created to check whether a tweet, news article, or search query contained a specific word, such as "cocaine" or "xtc." This process of topic feature extraction was essential in identifying and isolating the data that was relevant to the research question and in understanding the prevalence and significance of specific topics in the data.
  \item Normalization: The extracted features were normalized, as the values were absolute, while the Google Trends data were relative. Normalization between 0 and 100 was performed, and this came in useful when performing statistical tests and visualizing the data in the dashboard. Normalizing the data allowed for comparison and analysis of data from different sources, as the data was all expressed in the same unit.
\end{itemize}

Overall, the pre-processing of the Google Trends, Twitter, and Dutch news data was essential in ensuring that the data was in a suitable format for analysis. The conversion of dates, calculation of week numbers, topic feature extraction, and normalization were critical in making the data easier to analyze and interpret, and in ensuring that the results obtained from the analysis were accurate and reliable.

In addition,other pre-processing techniques such as sentiment analysis, locational feature extraction, word2vec synonym extraction and cleaning function have also been created. These functions could be used in future work to extend the research, make it more precise or compute detailed micro-level information.

\subsubsection{Data privacy}

In conducting the data collection and pre-processing of event-based drug popularity on online data resources, data privacy was a significant consideration. All data sources used were subjected to ethical and privacy considerations to ensure that all personal information was protected and that the data was collected, processed, and used in a responsible and ethical manner.

It was ensured that all data sources used were publicly available and did not contain any sensitive or personal information. The Google trends data was collected via Pytrends, in which personal information is already anonymized. For example, Google Trends data did not contain any information on the users who did the searches, but only the location where the search happened. The Twitter data was obtained using the SNscraper, which only collected data from public profiles and ensured that the data collected did not contain any personal information. For example, username and tags were removed.

In addition, appropriate measures were taken to protect the privacy of individuals and organizations that provided the data. For instance, all data was de-identified to remove any personal information that could be used to identify individuals or organizations.

\subsection {Statistical methods}

\subsubsection{Hypothesis}

To identify event-based drug popularity at events the following hypothesis was formulated:
There is a significant increase in the metrics (number of searches, -tweets and -news articles) of each data source during the weeks of the events known for increased drug usage. 

\subsubsection{Computing peaks per regular week}

A metric was constructed in order to translate the hypothesis into a test design. The data of each data source was aggregated weekly to represent the counts per searches, tweets and news articles respectively. 

The aggregated data was ordered chronologically and plotted on a line plot. The peaks in the line plot were found using the method signal.findpeaks from Pythons library Scipy [reference]. This step provided evidence on the weeks where there is an increase in the representative metric in relation to the other weeks in the considered time range (2014-2022). 

The data with labeled peaks was stored in a dataframe in the form of a Bool column coupled with a column for the corresponding week number. For instance if there was a peak in Week 2 2022, the Bool column would be populated with True, else it would be populated with False. With that each week number was represented with 9 rows, 1 for each year between 2014 and 2022. 
Finally, a proportion was calculated for each week representing the percentage of rows with “True” value out of the 9 rows for each week. With that, each week was represented with a number representing the percentage of years where a peak was observed for that particular week [Table with example from slides - Appendix]. 

\subsubsection{Computing peaks per event week}

The above steps were performed again in order to represent the percentage of peaks for the weeks of the events with increased drug usage (King’s day, ADE, Pride, Lowlands and New Year’s Eve). This was needed because most of these events do not always fall in the same week of the year.

For every year the dates of the events were identified. This was necessary because some of the events happen on a different date every year (e.g. ADE, Pride) [Dates per event - Appendix]. Based on the event dates, the week in which each event happened were identified. [Appendix]. Based on the data extracted from the line plot labeled with peaks, the years in which a peak happened per week of event were identified. Similarly as in the peaks per week, the percentage of weeks in which a peak was observed were calculated.

\subsubsection{Test design}

A p-value approach proportion test [reference] was performed on the generated data in order to test the hypothesis above. The test was once for each combination of drug, event and data source. Below are the specifications of the test:

\begin{itemize}
  \item Null hypothesis (H0): The proportion of peaks observed in drug related event weeks is not significantly larger than the proportion of peaks in regular weeks.
  \item Alternative hypothesis (H1): The proportion of peaks observed in drug related event weeks is significantly larger than the proportion of peaks in regular weeks.
  \item Level of significance: To determine level of significance, an alpha of 0.05 was used. This implies 5\% risk of concluding that there is a larger proportion when there in reality that is not the case.
  \item Test statistic (z): The test statistic was calculated based on the sample proportion and population proportion  [formula - Appendix], which is suitable for a proportion test. 
  \item Accepting the alternative hypothesis: The p-value was calculated from the test statistic. A p-value of less than 0.05 (corresponding to the significance level) indicated evidence to reject the null hypothesis and accept the alternative hypothesis, and hence prove that the proportion of peaks during event weeks is indeed larger than regular weeks.

\end{itemize}

\subsection{Visualization methods}

 \textit{GIVE INTRODUCTION ABOUT:  interface (visualization/controllability). Dashboard interface etc. Then move on to user requirements.}

These user requirements are derived from the case description provided by the client as well as feedback from the stakeholder during the initial ideation and prototyping phase of the project and where further narrowed down during the project design workshops using the MoSCoW priorization method. The term 'user' in the requirements refers to two specific types of similar target audiences that will make use of our prototype, police agents who want to explore the dataset to gain insights and data analysts who want to filter and compare our datasets.

 \begin{enumerate}
   \item (M) The user must be able to use the prototype on a personal computer and interface with a screen
   \item (M) The user must be able to filter the datasets to compare different years of data
   \item (M) The user must be able to filter the datasets to compare different types of drugs
   \item (M) The user must be able to overlay multiple datasets and trend lines on top of eachother
   \item (M) The system uses open-source software and not locked-in corporate data tools
   \item (M) The system has a user-friendly visual design and interaction design
   \item (S) The user should create an account to store specific and personalized filters
   \item (S) The user should be able to download the raw datasets in specific file formats
   \item (S) The user should be able to navigate between different overviews showing corresponding data
   \item (C) The user could upload their own dataset and sources of specfic drugs and news sources
   \item (C) The system uses real-time up-to-date API data
 \end{enumerate}

 \subsection{Collaboration}

 \textit{WRITE ABOUT: at a meta-level, team management. Something about the GitHub Org, Trello team management.}